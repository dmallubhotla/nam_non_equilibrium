\documentclass{article}

%other packages
\usepackage{amsmath}
\usepackage{amssymb}
\usepackage{physics}

\usepackage[
	style=phys, articletitle=false, biblabel=brackets, chaptertitle=false, pageranges=false, url=true
]{biblatex}

\usepackage{graphicx}
\usepackage{todonotes}
\usepackage{siunitx}

\usepackage[compat=1.0.0]{tikz-feynman}

\usepackage{cleveref}

\title{Notes on non-equilibrium Nam}

\addbibresource{./bibliography.bib}

\graphicspath{{./figures/}}

\newcommand{\pf}{p_{\mathrm{F}}}
\newcommand{\vf}{p_{\mathrm{F}}}
\newcommand{\corr}{\mu^\ast}
\DeclareMathOperator{\sgn}{sgn}

\begin{document}

\maketitle

\begin{figure}[htp]
	\centering
	\feynmandiagram [large, horizontal=a to b] {
		i1 -- [plain, momentum'={$p_1, \alpha$}] a,
		b -- [plain, momentum'={$p_3, \gamma$}] f3,
		i2 -- [plain, momentum={$p_2, \beta$}] a,
		b -- [plain, momentum={$p_4, \delta$}] f4,
		a -- [plain, quarter left, momentum={$p_1 + p_2 - k, \xi$}] b -- [plain, quarter left, rmomentum={$k, \sigma$}] a,
	};
	\caption{Test Feynman diagram} \label{fig:diagram}
\end{figure}

We will follow the derivation for the gap equation in AGD\cite{AGD}, in order to ensure that we can reproduce the Owen-Scalapino derivation~\cite{OwenScalapino}.
Unlike the Owen-Scalapino paper, which largely uses the operator formalism in the BCS paper~\cite{BCS}, I will try to rewrite-this derivation using the diagrammatic formalism in AGD.
This has the advantage of being more easily to relate to Nam and others (specifically the derivation summarised in the Nam paper~\cite{Nam1967}).

We start by looking at the corrections to the vertex function through the diagram in \cref{fig:diagram}.
With bare vertex function
\begin{equation}
	\Gamma_{\alpha \beta,\gamma \delta}(p_1, p_2; p_3, p_4) = \lambda \left( \delta_{\alpha\gamma} \delta_{\beta\delta} - \delta_{\alpha\delta}\delta_{\beta\gamma} \right),
\end{equation}

we can write for \cref{fig:diagram}
\begin{align}
	& i \lambda^2 \left( \delta_{\alpha\xi} \delta_{\beta\sigma} - \delta_{\alpha\sigma}\delta_{\beta\xi} \right)\left( \delta_{\xi\gamma} \delta_{\sigma\delta} - \delta_{\xi\delta}\delta_{\sigma\gamma} \right)\int \frac{\dd[4]{k}}{\left(2\pi\right)^4} G(k) G(p_1 + p_2 - k) \\
	&= i \lambda^2 \left( \delta_{\alpha\xi} \delta_{\beta\sigma} - \delta_{\alpha\sigma}\delta_{\beta\xi} \right)\left( \delta_{\xi\gamma} \delta_{\sigma\delta} - \delta_{\xi\delta}\delta_{\sigma\gamma} \right)\int \frac{\dd[4]{k}}{\left(2\pi\right)^4} G(k) G(p_1 + p_2 - k) \\
	&= i \lambda^2 \left( \delta_{\alpha\xi} \delta_{\beta\sigma} \delta_{\xi\gamma} \delta_{\sigma\delta} -  \delta_{\alpha\xi} \delta_{\beta\sigma} \delta_{\xi\delta}\delta_{\sigma\gamma} - \delta_{\alpha\sigma}\delta_{\beta\xi} \delta_{\xi\gamma} \delta_{\sigma\delta} + \delta_{\alpha\sigma}\delta_{\beta\xi}  \delta_{\xi\delta}\delta_{\sigma\gamma} \right)\int \frac{\dd[4]{k}}{\left(2\pi\right)^4} G(k) G(p_1 + p_2 - k) \\
	&= i \lambda^2 \left(\delta_{\alpha\gamma} \delta_{\beta\delta} -  \delta_{\alpha\delta} \delta_{\beta \gamma} - \delta_{\alpha\delta}\delta_{\beta\gamma}  + \delta_{\alpha\gamma}\delta_{\beta\delta} \right)\int \frac{\dd[4]{k}}{\left(2\pi\right)^4} G(k) G(p_1 + p_2 - k) \\
	&= 2 i \lambda^2 \left(\delta_{\alpha\gamma} \delta_{\beta\delta} -  \delta_{\alpha\delta} \delta_{\beta \gamma}\right)\int \frac{\dd[4]{k}}{\left(2\pi\right)^4} G(k) G(p_1 + p_2 - k)
\end{align}

The numerical factors aren't overly important here;
AGD actually omit some of these, including the factor of $2$ that emerges when handling the spin functions.
But it turns out to be unimportant, so the spin factor could also just be written down from what we know the vertex function must carry.

The Green's function here is
\begin{equation}
	G(k_0, \vec{k}) = \frac{a}{k_0 - \epsilon(\vec{k}) + \mu + \corr + i \delta s(\vec{k})},
\end{equation}
where $s(p) \equiv \sgn(\abs{\vec{p}} - \pf)$ and $\delta \rightarrow 0_{+}$ and $\corr$ is the ``correction factor'' representing the injected quasiparticles introduced by Owen and Scalapino.
This is a qualitatively different interpretation from theirs, because we are modifying the \emph{poles} of our quasiparticle Green's functions, which represents the additional ease which with they can be excited.

We can now insert this, and see how that affects our result for our diagram:
\begin{align}
	&2 i \lambda^2 \left(\delta_{\alpha\gamma} \delta_{\beta\delta} -  \delta_{\alpha\delta} \delta_{\beta \gamma}\right)\int \frac{\dd[4]{k}}{\left(2\pi\right)^4} G(k) G(p_1 + p_2 - k) \\
	&= 2 i \lambda^2 \left(\delta_{\alpha\gamma} \delta_{\beta\delta} -  \delta_{\alpha\delta} \delta_{\beta \gamma}\right)\int \frac{\dd[4]{k}}{\left(2\pi\right)^4} I,
\end{align}
where
\begin{equation}
	I = \frac{a}{k_0 - \epsilon(\vec{k}) + \mu + \corr + i \delta s(\vec{k})} \frac{a}{\omega_1 + \omega_2 - k_0 - \epsilon(\vec{k}) + \mu + \corr + i \delta s(\vec{p_1} + \vec{p_2} - \vec{k})}
\end{equation}

Begin with the integral over $k_0$, by collecting terms in each denominator into $\eta_1$ and $\eta_2$ so that
\begin{align}
	& 2 i \lambda^2 \left(\delta_{\alpha\gamma} \delta_{\beta\delta} -  \delta_{\alpha\delta} \delta_{\beta \gamma}\right)\int \frac{\dd[4]{k}}{\left(2\pi\right)^4} \frac{a}{k_0 + \eta_1 + i \delta s(\vec{k})} \frac{a}{- k_0 + \eta_2 + i \delta s(\vec{p_1} + \vec{p_2} - \vec{k})} \\
	&= 2 i \lambda^2 \left(\delta_{\alpha\gamma} \delta_{\beta\delta} -  \delta_{\alpha\delta} \delta_{\beta \gamma}\right)\int \frac{\dd[3]{\vec{k}}}{\left(2\pi\right)^4} \int \dd{k_0} \frac{a}{k_0 + \eta_1 + i \delta s(\vec{k})} \frac{a}{ - k_0 + \eta_2 + i \delta s(\vec{p_1} + \vec{p_2} - \vec{k})} \\
\end{align}

The $k_0$ integral can be done in closed form;
if the $s$ functions in each denominator have different sign, the result is zero.
Otherwise, call their shared sign $\sigma$, which is $1$ if $\abs{\vec{k}} > p_\mathrm{F}$ and $\abs{\vec{p}_1 + \vec{p}_2 - \vec{k}} > p_\mathrm{F}$, and $-1$ if both inequalities are reversed.

Then the result of the $k_0$ integral is
\begin{align}
	-\sigma 2 i \pi \frac{a^2}{\eta_1 + \eta_2 + 2 i \delta \sigma},
\end{align}
so our diagram becomes
\begin{align}
	&= - 2 i \lambda^2 \left(\delta_{\alpha\gamma} \delta_{\beta\delta} -  \delta_{\alpha\delta} \delta_{\beta \gamma}\right)\int \frac{\dd[3]{\vec{k}}}{\left(2\pi\right)^4} \sigma 2 i \pi \frac{a^2}{\eta_1 + \eta_2 + 2 i \delta \sigma} \\
	&= 2 a^2 \lambda^2 \left(\delta_{\alpha\gamma} \delta_{\beta\delta} -  \delta_{\alpha\delta} \delta_{\beta \gamma}\right)\int \frac{\dd[3]{\vec{k}}}{\left(2\pi\right)^3} \sigma \frac{1}{\eta_1 + \eta_2 + 2 i \delta \sigma} \\
	&= 2 a^2 \lambda^2 \left(\delta_{\alpha\gamma} \delta_{\beta\delta} -  \delta_{\alpha\delta} \delta_{\beta \gamma}\right)\int \frac{\dd[3]{\vec{k}}}{\left(2\pi\right)^3} \sigma \frac{1}{ - \epsilon(\vec{k}) + \mu + \corr + \omega_1 + \omega_2 - \epsilon(\vec{k}) + \mu + \corr + 2 i \delta \sigma} \\
\end{align}

\section{Nam conductivity}
As earlier, we use dimensionless variables.
\begin{align}
	\xi &= \frac{\omega}{\Delta} \\
	\xi' &= \frac{\omega'}{\Delta} \\
	\nu &= \frac{1}{\tau \Delta} \\
	\kappa &= \frac{q \vf}{\Delta} \\
	t &= \frac{T}{\Delta} \\
	\sigma_0 &= \frac{n e^2}{m \Delta}
\end{align}

From Nam, we have
\begin{equation}
	\sigma(\kappa, \xi) = -i \frac{3 \sigma_0}{4} \frac{1}{\xi}\left[\int_{1 - \xi}^{1}\dd{\xi} \tanh(\frac{\xi + \xi' - \mu}{2 t}) I_1 + \int_{1}^{\infty} \dd{\xi'} \left( \tanh(\frac{\xi + \xi' - \mu}{2t}) I_1  - \tanh(\frac{\xi' - \mu}{2t})I_2 \right) \right]
\end{equation}
with
\begin{align}
	I_1 &= F(\kappa, \Re[\sqrt{(\xi + \xi')^2 - 1} - \sqrt{\xi'^2 - 1}]) (g + 1) \nonumber\\
	&\quad + F(\kappa, \Re[-\sqrt{(\xi + \xi')^2 - 1} - \sqrt{\xi'^2 - 1}]) (g - 1) \\
	I_2 &= F(\kappa, \Re[\sqrt{(\xi + \xi')^2 - 1} - \sqrt{\xi'^2 - 1}]) (g + 1) \nonumber\\
	&\quad + F(\kappa, \Re[\sqrt{(\xi + \xi')^2 - 1} + \sqrt{\xi'^2 - 1}]) (g - 1) \\
	F(\kappa, E) &= \frac{1}{\kappa} \left[2 S(E) + (1 - S(E)^2)\ln(\frac{S(E) + 1}{S(E) - 1})\right]  \\
	S(\kappa, E) &= \frac{1}{\kappa} \left(E - i \left(\Im[\sqrt{(\xi + \xi')^2 - 1} + \sqrt{\xi'^2 - 1}] + 2 \nu \right) \right) \\
	g  &= \frac{\xi' \left( \xi + \xi'\right) + 1}{\sqrt{\xi'^2 - 1}\sqrt{(\xi + \xi')^2 - 1}}
\end{align}


\printbibliography

\end{document}
