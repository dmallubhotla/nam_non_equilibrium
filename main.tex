\documentclass{article}

%other packages
\usepackage{amsmath}
\usepackage{amssymb}
\usepackage{physics}

\usepackage[
	style=phys, articletitle=false, biblabel=brackets, chaptertitle=false, pageranges=false, url=true
]{biblatex}

\usepackage{graphicx}
\usepackage{todonotes}
\usepackage{siunitx}

\title{Notes on non-equilibrium Nam}

\addbibresource{./bibliography.bib}

\graphicspath{{./figures/}}


\begin{document}

\maketitle

Our goal is to extend the Nam expression to incorporate non-equilibrium effects for arbitrary wave-vector.

\section{Derivation}
The critical expression for our purposes is equation 4.2 in Nam\cite{Nam1967}:
\begin{equation}
	I = \int \frac{\dd{\epsilon_k}}{2 \pi i} \int_0 \dd{\omega} f(\omega) \Tr{\mathrm{Greens\ function}}
\end{equation}
For this, Nam draws contours around the poles of the Green's function component, and does the integral.

From~\cite{Gao2008}, we can model the non-equilibrium components by replacing the Fermi distribution function $f(\omega)$ with $f(\omega - \mu)$, where $\mu$ represents the chemical potential of quasiparticles (which, using expressions in Gao, can be derived from a fixed quasiparticle density).
The manipulation of the Green's functions Nam carries out remain valid, and it remains only to actually evaluate those integrals.

\printbibliography

\end{document}
